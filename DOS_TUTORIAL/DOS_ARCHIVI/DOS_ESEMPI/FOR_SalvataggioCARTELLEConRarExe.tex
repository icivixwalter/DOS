\documentclass[10pt,a4paper]{book}
\usepackage[utf8]{inputenc}
\usepackage{amsmath}
\usepackage{amsfonts}
\usepackage{amssymb}
\usepackage{graphicx}
\begin{document}
	
	\subsection{FOR SALVATAGGIO CARTELLA + RAR.EXE + FILE LOG}
		\label{FOR:SALVATAGGI-DirectoryConRAR+LOG}{....CICLO FOR DI SALVATAGGI CON RAR.EXE + FILE LOG.TXT}
		Ciclo for di salvataggio DELLE SOLE CARTELLE e sottodirectory in esse contenute.
		Non sono ammessi salvataggi esterni. Utilizza il RAR.EXE ed ha in più la particolarita
		di costruire un FILE LOG.TXT che contiene le operazioni di salvataggio.
		\\
		Per utilizzarlo basta cambiare i parametri sopra in testata.
		\\
		Il file è stato denominato $Vittoria_N01_SALVATAGGI_CICLO_FOR+LOG.bat$ e vengono
		uniti i file cripta e decodifica per attivare il controllo a mano, entrambi denominati in\\
		$Vittoria_N70_CRIPTA.bat$\\
		$Vittoria_N70_DECODIFICA.bat$\\
		La procedura esegue il salvataggio in una directori denominata XSALVATAGGI il cui nome può
		essere cambiato nei parametri, venendo creata quando essa non esiste. In questa directory
		vengono salvati prima i file .rar che alla fine del processo vengono criptati in .GE614 
		unitamente al file log del dos .txt
		\\
		ESEMPIO:\\
		
		\detokenize{
			@REM  ***************************************************************
			
			@REM      file batch : Salva-C rar
			@REM      directory = c:\Casa\CDM\Vittoria\ ; \DOC  e \STAMPE
			@REM      file WinRAR
			
			@REM   ***************************************************************
			
			
			@REM 					LE SOSTITUZIONI GENERALI
			@REM @@@@@@@@@@@@@@@@@@@@@@@@@@@@@@@@@@@@@@@@@@@@@@@@@@@@@@@@@@@@@@@@@@@@@@@@@@@@@@@@@@@@@@@@@@@@@@@@@@@@@@@@@@@@@@@@@@@@@@@@
			
			@REM  		Disco e codice Disco
			@REM .......................................................
			@REM -----> 	C:
			@REM -----> 	_C
			@REM .......................................................
			
			
			@REM		directory y dove archiviare i dati
			@REM .......................................................
			@REM ----->	C:\CASA\CDM\VITTORIA\
			@REM .......................................................
			
			
			@REM 		Il nome di tutti i file di archivio dei dati
			@REM .......................................................
			@REM ----->	Vittoria_
			
			
			@REM 		Il file di archivio dei dati
			@REM .......................................................
			@REM ----->	*.GE614
			
			
			@REM 		I parametri relativi al file log + msg + path
			@REM .......................................................
			@REM ----->	PATH_S=c:\Casa\CDM\Vittoria\XX_SALVATAGGI\
			@REM ----->	FILE_LOG_S=VittoriaFileLOG.txt
			@REM ----->	MSG_S=MSG 01 - SALVATAGGIO CARTELLE VITTORIA CON CICLO FOR E RAR.EXE
			
			
			
			
			@REM 					LE SOSTITUZIONI GENERALI *** FINE ***
			@REM @@@@@@@@@@@@@@@@@@@@@@@@@@@@@@@@@@@@@@@@@@@@@@@@@@@@@@@@@@@@@@@@@@@@@@@@@@@@@@@@@@@@@@@@@@@@@@@@@@@@@@@@@@@@@@@@@@@@@@@@
			
			
			
			
			ECHO OFF
			CLS
			
			
			@REM 			SALVATAGGIO DATI
			REM *************************************************************************************************************************
			
			@REM			MESSAGGIO OPERAZIONE
			@REM -----------------------------------------------------------------------------------------
			echo  "============================================================="
			echo.				INIZIO
			echo.	MSG - SALVATAGGIO DATI - Vittoria_N05_ARCHIVI_BAT.GE614
			echo.  
			echo.
			echo  "============================================================="
			
			@REM			MESSAGGIO OPERAZIONE *** FINE ***
			@REM -----------------------------------------------------------------------------------------
			
			
			@REM			OPERAZIONE DI SALVATAGGIO
			@REM -----------------------------------------------------------------------------------------
			
			@REM RESET VARIABILI --- Imposto la path di criptagio e decriptaggio
			@SET PATH_S=c:\Casa\CDM\Vittoria\XX_SALVATAGGI\
			@SET FILE_LOG_S=VittoriaFileLOG.txt
			@SET MSG_S=MSG 01 - SALVATAGGIO CARTELLE VITTORIA CON CICLO FOR E RAR.EXE
			
			
			
			:----------------------------INIZIO = Creo il file
			echo %MSG_S%											> %PATH_S%%FILE_LOG_S%
			
			:----------------------------CONTROLLO se esiste la directory 
			
			
			@SET MSG_S= MSG 02 - "if exist "SE ESISTE LA cartella XX_SALVATAGGI, VIENE ATTIVATA LA DECODIFICA DEI FILE .GE614 IN RAR
			echo .											>> %PATH_S%%FILE_LOG_S%
			echo .CONTROLLO DI ESISTENZA DIRECTORY							>> %PATH_S%%FILE_LOG_S%
			echo "%cd%\XX_SALVATAGGI\"								>> %PATH_S%%FILE_LOG_S%
			echo ."..............................................................." 		>> %PATH_S%%FILE_LOG_S%
			
			echo .											>> %PATH_S%%FILE_LOG_S%
			
			@REM RESET VARIABILE INTERNA deve stare all'esterno della if altrimenti deve essere una variabile a espansione ritardata
			@SET CODICE_S=" CODICE -> CONTROLLO ESISTENZA DIRECTORY - > "
			if exist "%cd%\XX_SALVATAGGI\" (
			
			ECHO 01 INIZIO %CODICE_S%
			
			echo .										>> %PATH_S%%FILE_LOG_S%
			echo .										>> %PATH_S%%FILE_LOG_S%
			echo. "======================================================="			>> %PATH_S%%FILE_LOG_S%
			echo   OPERAZIONE %MSG_S%	" *****INIZIO****** "				>> %PATH_S%%FILE_LOG_S%
			echo. "======================================================="			>> %PATH_S%%FILE_LOG_S%
			echo .										>> %PATH_S%%FILE_LOG_S%
			echo %MSG_S%									>> %PATH_S%%FILE_LOG_S%
			echo .										>> %PATH_S%%FILE_LOG_S%
			echo .										>> %PATH_S%%FILE_LOG_S%
			
			echo "................CREO IL FILE VITTORIA_LOG..........."			>> %PATH_S%%FILE_LOG_S%
			echo .										>> %PATH_S%%FILE_LOG_S%
			
			
			ECHO 02 "OPERAZIONE %MSG_S% "	%CODICE_S%	 
			@REM//DECODIFICA I FILE G614 IN RAR
			echo.										>> %PATH_S%%FILE_LOG_S%
			echo. "OPERAZIONE %MSG_S%"							>> %PATH_S%%FILE_LOG_S%
			echo. "MSG --->DECODIFICA I FILE DA .GE614 IN .RAR"				>> %PATH_S%%FILE_LOG_S%
			echo "-------------------------------------------------------"  		>> %PATH_S%%FILE_LOG_S%
			echo.										>> %PATH_S%%FILE_LOG_S%
			echo. "decodifico i file .GE614 in .RAR con rename"				>> %PATH_S%%FILE_LOG_S%
			echo.										>> %PATH_S%%FILE_LOG_S%
			
			
			ECHO  03 STAMPO ELENCO DA DECODIFICARE *.GE614 %CODICE_S%
			echo.
			echo. " stampo elenco DIR *.GE614 da decodificare"				>> %PATH_S%%FILE_LOG_S%
			echo. " ..........................................."				>> %PATH_S%%FILE_LOG_S%
			echo.										>> %PATH_S%%FILE_LOG_S%
			
			
			ECHO  04 Stampo il contenuto della directory nel file.txt *.GE614 %CODICE_S%
			DIR %PATH_S%*.GE614 		@rem faccio il controllo con dir					
			DIR %PATH_S%*.GE614 								>> %PATH_S%%FILE_LOG_S%
			
			
			
			ECHO 04 APPLICO IL COMANDO RENAME  %CODICE_S%
			echo.										>> %PATH_S%%FILE_LOG_S%
			echo. "DECODIFICO dei file da *.GE614 A .RAR"					>> %PATH_S%%FILE_LOG_S%
			echo.										>> %PATH_S%%FILE_LOG_S%
			@RENAME %PATH_S%*.GE614  *.RAR
			
			
			
			echo.
			echo. " Sopra l'elenco dei file CODIFICATI IN .GE64"				>> %PATH_S%%FILE_LOG_S%
			echo. " ..........................................."				>> %PATH_S%%FILE_LOG_S%
			echo.										>> %PATH_S%%FILE_LOG_S%
			
			echo.										>> %PATH_S%%FILE_LOG_S%
			echo. "======================================================="			>> %PATH_S%%FILE_LOG_S%
			echo. "                  *****fine ******                     "			>> %PATH_S%%FILE_LOG_S%
			echo. "======================================================="			>> %PATH_S%%FILE_LOG_S%
			echo.										>> %PATH_S%%FILE_LOG_S%
			
			@REM STOP PROCEDURA ->  PAUSE echo. "PAUSE in questa POSIZIONE ------->" 
			
			) ELSE (
			
			@SET MSG_S= MSG 03 - SE NON ESISTE LA DIRECTORY  IL DECRITPAGGIO NON VIENE ESEGUITO.
			
			echo.										>> %PATH_S%%FILE_LOG_S%
			echo.										>> %PATH_S%%FILE_LOG_S%
			echo. "======================================================="			>> %PATH_S%%FILE_LOG_S%
			echo   OPERAZIONE %MSG_S%	" *****INIZIO****** "				>> %PATH_S%%FILE_LOG_S%
			echo. "======================================================="			>> %PATH_S%%FILE_LOG_S%
			echo.										>> %PATH_S%%FILE_LOG_S%
			echo.										>> %PATH_S%%FILE_LOG_S%
			
			echo. "MSG --->DECODIFICA I FILE DA .GE614"					>> %PATH_S%%FILE_LOG_S%
			echo. "For di salvataggi delle CARTELLE con RAR.EXE"				>> %PATH_S%%FILE_LOG_S%
			echo.										>> %PATH_S%%FILE_LOG_S%
			echo.										>> %PATH_S%%FILE_LOG_S%
			echo. "======================================================="			>> %PATH_S%%FILE_LOG_S%
			echo. "                  *****fine ******                     "			>> %PATH_S%%FILE_LOG_S%
			echo. "======================================================="			>> %PATH_S%%FILE_LOG_S%
			echo.										>> %PATH_S%%FILE_LOG_S%
			
			)
			
			
			
			
			echo off
			
			:------------------------------------------IMPOSTO IL MESSAGGIO + TITOLO + FINE
			@SET MSG_S= MSG 05 - SE NON ESISTE LA DIRECTORY  VIENE CREATA LA DIRECTORY "%cd%\XX_SALVATAGGI\"
			echo.										>> %PATH_S%%FILE_LOG_S%
			echo.										>> %PATH_S%%FILE_LOG_S%
			echo. "======================================================="			>> %PATH_S%%FILE_LOG_S%
			echo  " 		*****INIZIO****** 		      "			>> %PATH_S%%FILE_LOG_S%
			echo   OPERAZIONE %MSG_S%							>> %PATH_S%%FILE_LOG_S%
			echo. "======================================================="			>> %PATH_S%%FILE_LOG_S%
			echo.										>> %PATH_S%%FILE_LOG_S%
			echo.										>> %PATH_S%%FILE_LOG_S%
			
			:----------------------------SET DELLE VARIABILI PATH DI PARTENZA ED ARRIVO
			
			@REM Se non esiste la cartella XX_SALVATAGGI, la crea.
			if not exist "%cd%\XX_SALVATAGGI\" (
			echo Crea la cartella XX_SALVATAGGI
			mkdir %cd%\XX_SALVATAGGI
			
			echo.										>> %PATH_S%%FILE_LOG_S%
			echo.										>> %PATH_S%%FILE_LOG_S%
			echo. "======================================================="			>> %PATH_S%%FILE_LOG_S%
			echo  " CREATA LA DIRECTORY 				      "			>> %PATH_S%%FILE_LOG_S%
			echo   "%cd%\XX_SALVATAGGI\" 						>> %PATH_S%%FILE_LOG_S%
			echo. "======================================================="			>> %PATH_S%%FILE_LOG_S%
			echo.										>> %PATH_S%%FILE_LOG_S%
			echo.										>> %PATH_S%%FILE_LOG_S%
			echo.										>> %PATH_S%%FILE_LOG_S%
			echo. "======================================================="			>> %PATH_S%%FILE_LOG_S%
			echo. "                  *****fine ******                     "			>> %PATH_S%%FILE_LOG_S%
			echo. "======================================================="			>> %PATH_S%%FILE_LOG_S%
			echo.										>> %PATH_S%%FILE_LOG_S%
			
			)
			
			set PATH_ARRIVO_s="%cd%\XX_SALVATAGGI\"
			
			echo. 
			echo controllo della path di ARRIVO:
			echo "PATH ARRIVO: " %PATH_ARRIVO_s%
			
			
			:----------------------CICLO_FOR= Ciclo For su ogni riga del comando DIR (senza dettagli con le sottocartelle)
			
			
			@REM L'unico modo per salvare SOLO le cartelle e non i file. Se vuoi salvare anche i file RIMUOVI /A:d
			for /f "tokens=*" %%G in ('dir /B /A:d %cd%') do (
			
			@REM Per ogni cartella diversa da XX_SALVATAGGI...
			echo controllo directory in esame %%G
			echo controllo path da escludere %PATH_ARRIVO_s%
			
			
			@REM //NOTE DI FUNZIONAMENTO
			@REM //============================================================================//
			@REM attenzione per il salvataggio sono stati utilizzati questi parametri:
			@REM dir 	/b = elenca il contenuto della cartella corrente senza dettagli
			@REM 		/s = Mostra il percorso completo di ogni file/cartella E DELLE SOTTOCARTELLE!!!!
			@REM 		/a:d = include SOLO LE CARTELLE (escluso perche altrimenti salvava ogni singolo
			@REM 		file della sottocartella in un zip)
			@REM 		COMANDO rar
			@REM rar.exe	-ep1= esclude la path nel .zip e inserisce solo il nome della sottocartella nell'archivio.	
			
			
			IF NOT %%~nG==XX_SALVATAGGI (
			echo --------------  Estraggo "%%~nG" e salvo nella path di arrivo.
			@REM non ci deve essere spazio altrimenti il nome del file viene con lo spazio SI=%PATH_ARRIVO_s%%%~nG ... NO=%PATH_ARRIVO_s% %%~nG
			C:\CASA\Rar.exe a -ep1 "%PATH_ARRIVO_s%%%~nG" "%%~fG"
			
			@SET MSG_S= MSG 06 - ELENCO FILE INSERITI NEL RAR
			echo.									>> %PATH_S%%FILE_LOG_S%
			echo.									>> %PATH_S%%FILE_LOG_S%
			echo. %MSG_S%								>> %PATH_S%%FILE_LOG_S%		
			echo.									>> %PATH_S%%FILE_LOG_S%
			DIR "%PATH_ARRIVO_s%%%~nG" "%%~fG"					>> %PATH_S%%FILE_LOG_S%
			
			
			
			
			)
			)
			
			
			
			
			@REM @@@@@@@@@@@@@@@@@@@@@@@@@@@@@@@@@@@@@@@@@@@@@@@@@@@@@@@@@@@@@@@@@@@@@@@@@@@@@@@@@@@@@@@@
			@REM 			TITOLO OPERAZIONE
			@REM
			@SET CODICE_S= SALVATAGGIO DEI FILE .BAT + TXT
			@REM ........................................................................................
			@REM ..........................................................
			@REM NOTE = set deve stare al di fuori della if altrimenti 
			@REM non funziona, forse è dovuto alla espansione ritardata.
			@REM
			@REM ........................................................................................
			@REM
			echo.
			echo. MESSAGGIO %CODICE_S%
			echo. "-------------------------------------------------------"	
			echo.	Eseguo il salvataggio dei file .bat .txt									
			echo.	nella directory di salvataggio									
			echo. "-------------------------------------------------------"	
			echo.										
			@REM --> PAUSE
			
			@REM CODICE...
			C:\CASA\Rar.exe a -ep1 %PATH_ARRIVO_s%FILE_BAT.RAR *.BAT *.TXT
			
			
			@REM			FINE
			@REM @@@@@@@@@@@@@@@@@@@@@@@@@@@@@@@@@@@@@@@@@@@@@@@@@@@@@@@@@@@@@@@@@@@@@@@@@@@@@@@@@@@@@@@@
			
			
			
			
			
			
			echo.
			echo. MESSAGGIO %CODICE_S%
			echo. "-------------------------------------------------------"	
			echo.	ATTIVITA									
			echo.	ATTIVITA									
			echo. "-------------------------------------------------------"	
			echo.										
			
			
			
			
			@REM//CRIPTA I FILE RAR IN G614
			@SET MSG_S= MSG 09 - CRIPTO I FILE .RAR IN .GE614
			echo. 								>> %PATH_S%%FILE_LOG_S%
			echo. "trasformo i file .RAR in file CRIPTATI .GE614"		>> %PATH_S%%FILE_LOG_S%
			echo.								>> %PATH_S%%FILE_LOG_S%
			echo. %MSG_S%							>> %PATH_S%%FILE_LOG_S%		
			echo.								>> %PATH_S%%FILE_LOG_S%
			echo "ELENCO FILE .RAR TRASFORMATI DA RAR IN .GE61"		>> %PATH_S%%FILE_LOG_S%
			echo.								>> %PATH_S%%FILE_LOG_S%
			DIR *.RAR							>> %PATH_S%%FILE_LOG_S%
			@REM//CRIPTA i file .rar .zip in GE614
			@RENAME %PATH_S%*.RAR *.GE614 
			
			
			@REM FINE 			eof
			
			@REM			OPERAZIONE DI SALVATAGGIO *** FINE ***
			@REM -----------------------------------------------------------------------------------------
			
			
			@REM			SOSPENSIONE
			@REM -----------------------------------------------------------------------------------------
			
			@REM 01)
			@REM sospensione per 1 secondo con crononometro 
			@REM timeout /t 2 /nobreak > NUL
			@REM ---->	TIMEOUT /T 2 /NOBREAK
			
			EXIT
			@REM			SOSPENSIONE *** FINE ***
			@REM -----------------------------------------------------------------------------------------
			
			
			@REM 			SALVATAGGIO DATI *** FINE ***
			@REM *************************************************************************************************************************
			
			
			
			
			
			***************************************************************************************************************************
			goto Modelli
			
			Pe attribruire un titolo di inizio e fine utilizzare questo schema nel quale
			set imposta la variabile del messaggio, poi viene scritta una intestazione, un corpo
			delle attivia ed una fine:
			
			:------------------------------------------IMPOSTO IL MESSAGGIO + TITOLO + FINE
			@SET MSG_S= MSG 05 - SE NON ESISTE LA DIRECTORY  VIENE CREATA LA DIRECTORY "%cd%\XX_SALVATAGGI\"
			echo.										>> %PATH_S%%FILE_LOG_S%
			echo.										>> %PATH_S%%FILE_LOG_S%
			echo. "======================================================="			>> %PATH_S%%FILE_LOG_S%
			echo  " 		*****INIZIO****** 		      "			>> %PATH_S%%FILE_LOG_S%
			echo   OPERAZIONE %MSG_S%							>> %PATH_S%%FILE_LOG_S%
			echo. "======================================================="			>> %PATH_S%%FILE_LOG_S%
			echo. ... operazioni								>> %PATH_S%%FILE_LOG_S%
			echo.										>> %PATH_S%%FILE_LOG_S%
			echo.										>> %PATH_S%%FILE_LOG_S%
			echo.										>> %PATH_S%%FILE_LOG_S%
			echo.										>> %PATH_S%%FILE_LOG_S%
			echo. "======================================================="			>> %PATH_S%%FILE_LOG_S%
			echo. "                  *****fine ******                     "			>> %PATH_S%%FILE_LOG_S%
			echo. "======================================================="			>> %PATH_S%%FILE_LOG_S%
			echo.										>> %PATH_S%%FILE_LOG_S%
			
			
			
			
			
			
			
			
			IL MESSAGGIO DELLA POSIZIONE = 	1 Imposto il codice del messaggio con set
			2 imposto echo + codice messaggio
			
			Il messaggio della posizione serve per individuare con pause la posizione del codice.				
			
			@REM @@@@@@@@@@@@@@@@@@@@@@@@@@@@@@@@@@@@@@@@@@@@@@@@@@@@@@@@@@@@@@@@@@@@@@@@@@@@@@@@@@@@@@@@
			@REM 				TITOLO OPERAZIONE
			@REM
			@SET CODICE_S= SALVATAGGIO DEI FILE .BAT + TXT
			@REM ........................................................................................
			@REM ..........................................................
			@REM NOTE = set deve stare al di fuori della if altrimenti 
			@REM non funziona, forse è dovuto alla espansione ritardata. Il titolo è corpo del messaggio
			@REM ed indica la posizione delle attività, in più ci sono 2 pause per il blocco del codice.	
			@REM
			@REM ........................................................................................
			@REM
			echo.
			echo. MESSAGGIO %CODICE_S%
			echo. "-------------------------------------------------------"	
			echo.	ATTIVITA									
			echo.	ATTIVITA									
			echo. "-------------------------------------------------------"	
			echo.										
			@REM --> PAUSE
			
			@REM CODICE...
			...			
			@REM --> PAUSE
			@REM			FINE
			@REM @@@@@@@@@@@@@@@@@@@@@@@@@@@@@@@@@@@@@@@@@@@@@@@@@@@@@@@@@@@@@@@@@@@@@@@@@@@@@@@@@@@@@@@@
			
			
			
			
			:Modelli
			
		
		}
		
	\subsection{FOR Ciclo di SALVAGGIO CARTELLE CON RAR.EXE}
	\label{FOR:SALVATAGGI-DirectoryConRAR}{....CICLO FOR DI SALVATAGGI CON RAR.EXE}
	Esempio comprimi le directory e sottodirectory con un CICLO FOR che costruisce
	una directory XAA\_SALVATAGGI nel quale comprimere con il comando rar tutte le
	directory. E' importante applicare il ciclo solo sulle directory create in quella
	corrente. Non salva i file esterni e se non ci sono directory già create non salva nulla.
	\\
	ESEMPIO:\\
	
	\detokenize{
		
		echo off
		
		:----------------------------SET DELLE VARIABILI PATH DI PARTENZA ED ARRIVO
		
		@REM Se non esiste la cartella XX_SALVATAGGI, la crea.
		if not exist "%cd%\XX_SALVATAGGI\" (
		echo Crea la cartella XX_SALVATAGGI
		mkdir %cd%\XX_SALVATAGGI
		)
		
		set PATH_ARRIVO_s="%cd%\XX_SALVATAGGI\"
		
		echo. 
		echo controllo della path di ARRIVO:
		echo "PATH ARRIVO: " %PATH_ARRIVO_s%
		
		
		:----------------------CICLO_FOR= Ciclo For su ogni riga del comando DIR (senza dettagli con le sottocartelle)
		echo Prova DIR
		
		@REM L'unico modo per salvare SOLO le cartelle e non i file. Se vuoi salvare anche i file RIMUOVI /A:d
		for /f "tokens=*" %%G in ('dir /B /A:d %cd%') do (
		
		@REM Per ogni cartella diversa da XX_SALVATAGGI...
		echo controllo directory in esame %%G
		echo controllo path da escludere %PATH_ARRIVO_s%
		IF NOT %%~nG==XX_SALVATAGGI (
		echo --------------  Estraggo "%%~nG" e salvo nella path di arrivo.
		@REM non ci deve essere spazio altrimenti il nome del file viene con lo spazio SI=%PATH_ARRIVO_s%%%~nG ... NO=%PATH_ARRIVO_s% %%~nG
		C:\CASA\Rar.exe a -ep1 "%PATH_ARRIVO_s%%%~nG" "%%~fG"
		)
		)
		eof
		
		
		@REM //NOTE DI FUNZIONAMENTO
		@REM //============================================================================//
		@REM attenzione per il salvataggio sono stati utilizzati questi parametri:
		@REM dir 	/b = elenca il contenuto della cartella corrente senza dettagli
		@REM 	/s = Mostra il percorso completo di ogni file/cartella E DELLE SOTTOCARTELLE!!!!
		@REM 	/a:d = include SOLO LE CARTELLE (escluso perche altrimenti salvava ogni singolo
		@REM 	file della sottocartella in un zip)
		@REM COMANDO rar
		@REM rar.exe	-ep1= esclude la path nel .zip e inserisce solo il nome della sottocartella nell'archivio.	
		
		
		
		
		
	}
	
\end{document}